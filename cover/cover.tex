%!TEX root=../thesis.tex
%\newpage
%\setcounter{page}{1}
\begin{center}
\begin{figure}
\centering
\epsfig{file=cover/EscudoUN.eps,scale=1}
\end{figure}
\thispagestyle{empty} \vspace*{2.0cm} \textbf{\huge
Sistemas Dinámicos Planos}\\[6.0cm]
\Large\textbf{Jorge Alberto Torres Henao}\\[6.0cm]
\small Universidad Nacional de Colombia\\
Facultad de Ciencias, Escuela de Matemáticas\\
Medellín, Colombia\\
2013\\
\end{center}

\newpage{\pagestyle{empty}\cleardoublepage}

\newpage
\begin{center}
\thispagestyle{empty} \vspace*{0cm} \textbf{\huge
Sistemas Dinámicos Planos}\\[3.0cm]
\Large\textbf{Jorge Alberto Torres Henao}\\[3.0cm]
\small Tesis presentada como requisito parcial para optar al
t\'{\i}tulo de:\\
\textbf{Magíster en Ciencias - Matemáticas}\\[2.5cm]
Director:\\
Ph.D. Carlos Enrique Mejía Salazar\\[4.5cm]
Universidad Nacional de Colombia\\
Facultad de Ciencias, Escuela de Matemáticas\\
Medellín, Colombia\\
2013\\
\end{center}

\newpage{\pagestyle{empty}\cleardoublepage}

\thispagestyle{empty} \textbf{}\normalsize
\\\\\\%

\null
\vfill
\begin{flushright}
\begin{minipage}{10.5cm}
    \noindent
    {\small
Parcialmente apoyado por Colciencias proyecto 1118-48925120 y por la Vicerrectoría de Investigación de la Universidad Nacional de Colombia, proyectos con código QUIPU 20101009545 y 201010011107.		
	}
\end{minipage}
\end{flushright}

\newpage{\pagestyle{empty}\cleardoublepage}


%
\newpage
\textbf{\LARGE Resumen}
\addcontentsline{toc}{chapter}{\numberline{}Resumen}\\\\
En el documento se hace una presentación de los sistemas dinámicos planos, en particular los continuos provenientes de ecuaciones diferenciales ordinarias de segundo orden o de sistemas de ecuaciones diferenciales simultáneas de primer orden. Se introduce la teoría con ejemplos cuidadosamente seleccionados que hacen especial énfasis en las propiedades topológicas de las soluciones y se dedica toda una sección a la obtención de una demostración del Teorema de Poincaré-Bendixson. La parte teórica finaliza con la revisión de algunas bifurcaciones 1D y 2D.\\
El trabajo concluye con la presentación de un software libre desarrollado por el autor que sirve de apoyo para la creación de diagramas de fase de cualquier sistema dinámico plano.
\\

\textbf{\small Palabras clave: dynamical system, planar system, orbit, phase portrait, limit cycle, bifurcation}.\\
%
%A continuaci\'{o}n se presentan algunos ejemplos de tesauros que se pueden consultar para asignar las palabras clave, seg\'{u}n el \'{a}rea tem\'{a}tica:\\
%
%\textbf{Artes}: AAT: Art y Architecture Thesaurus.
%
%\textbf{Ciencias agropecuarias}: 1) Agrovoc: Multilingual Agricultural Thesaurus - F.A.O. y 2)GEMET: General Multilingual Environmental Thesaurus.
%
%\textbf{Ciencias sociales y humanas}: 1) Tesauro de la UNESCO y 2) Population Multilingual Thesaurus.
%
%\textbf{Ciencia y tecnolog\'{\i}a}: 1) Astronomy Thesaurus Index. 2) Life Sciences Thesaurus, 3) Subject Vocabulary, Chemical Abstracts Service y 4) InterWATER: Tesauro de IRC - Centro Internacional de Agua Potable y Saneamiento.
%
%\textbf{Tecnolog\'{\i}as y ciencias m\'{e}dicas}: 1) MeSH: Medical Subject Headings (National Library of Medicine's USA) y 2) DECS: Descriptores en ciencias de la Salud (Biblioteca Regional de Medicina BIREME-OPS).
%
%\textbf{Multidisciplinarias}: 1) LEMB - Listas de Encabezamientos de Materia y 2) LCSH- Library of Congress Subject Headings.\\
%
%Tambi\'{e}n se pueden encontrar listas de temas y palabras claves, consultando las distintas bases de datos disponibles a trav\'{e}s del Portal del Sistema Nacional de Bibliotecas\footnote{ver: www.sinab.unal.edu.co}, en la secci\'{o}n "Recursos bibliogr\'{a}ficos" opci\'{o}n "Bases de datos".\\
%
%\textbf{\LARGE Abstract}\\\\
%Es el mismo resumen pero traducido al ingl\'{e}s. Se debe usar una extensi\'{o}n m\'{a}xima de 12 renglones. Al final del Abstract se deben traducir las anteriores palabras claves tomadas del texto (m\'{\i}nimo 3 y m\'{a}ximo 7 palabras), llamadas keywords. Es posible incluir el resumen en otro idioma diferente al espa\~{n}ol o al ingl\'{e}s, si se considera como importante dentro del tema tratado en la investigaci\'{o}n, por ejemplo: un trabajo dedicado a problemas ling\"{u}\'{\i}sticos del mandar\'{\i}n seguramente estar\'{\i}a mejor con un resumen en mandar\'{\i}n.\\[2.0cm]
%\textbf{\small Keywords: palabras clave en ingl\'{e}s(m\'{a}ximo 10 palabras, preferiblemente seleccionadas de las listas internacionales que permitan el indizado cruzado)}\\
