%!TEX root=../thesis.tex
\chapter{Introducción}
\label{cap:introduccion}

Los sistemas dinámicos aparecen al tratar de especificar mediante un modelo matemático procesos en los que es posible describir la dependencia en el tiempo de un punto en un espacio geométrico mediante la aplicación de una fórmula o ``regla''. Surgen, entonces, con naturalidad en virtualmente todas las áreas de la ciencia como pueden serlo la biología, química o física y existen también modelos relacionados con problemas netamente teóricos como en el caso de los autómatas celulares \cite{schiffautomata} o las figuras fractales en el plano complejo.

Esta definición tan amplia permite incluir dentro de la definición de sistemas dinámicos fenómenos tan dispares como el movimiento en un sistema mecánico (como un péndulo, por ejemplo) o el número de individuos de una población de peces en un lago en el tiempo; pasando inclusive por fenómenos relacionados con procesos químicos en los que hay intercambio de materia o la predicción del clima \cite{lorenz64}.

La clave para esta unificación se encuentra en el concepto de ``estado'' y ``regla de evolución'': un sistema, en un instante de tiempo dado, se encuentra en algún estado posible, representado generalmente como un punto en el espacio euclídeo $\R^n$. La regla de evolución del sistema es una regla fija (función) que determina el estado futuro de dicho punto. Estos términos se aclaran en la sección \ref{sec:conceptosbasicos}.

El caso que nos ocupa en la presente tesis es aquel en el cual el espacio de estados es un subconjunto del plano $\R^2$ y, en particular, cuando el sistema dinámico está definido por un fenómeno en el que intervienen ecuaciones diferenciales de segundo orden o sistemas de dos ecuaciones de primer orden. Aunque la reducción al caso $n = 2$ pareciera una simplificación sustancial, en realidad no lo es: la teoría de sistemas dinámicos bidimensionales es fundamental en la generalización a más dimensiones y es, además, exclusivamente en el plano en donde se observan comportamientos particularmente interesantes como el especificado por el Teorema de Poincaré-Bendixson (capítulo \ref{cap:poincarebendixson}). Además, si identificamos a $\C$ con el plano $\R^2$ es posible incluir también en este estudio los sistemas relacionados con iteraciones de polinomios con coeficientes complejos, área de la que surgen los conjuntos fractales.

El desarrollo teórico que se hace en adelante corresponde a los temas que serían tratados en un curso de nivel de posgrado sobre sistemas dinámicos. Se ha hecho particular énfasis en la escogencia de resultados y ejemplos que resulten ilustrativos.

Finalmente, la cuota de originalidad del presente trabajo está a cargo de DYNAMITE (capítulo \ref{chap:dynamite}), software matemático que se desarrolló desde cero para acompañar este trabajo \footnote{Las figuras en este trabajo que fueron generadas  utilizando DYNAMITE están especialmente marcadas con el símbolo $\clubsuit$.} y que puede ser utilizado como apoyo para el estudio de sistemas dinámicos planos en general. El código de DYNAMITE, además, está distribuido bajo una licencia libre, lo que permite su redistribución y uso sin costo alguno.