\documentclass{standalone}

\usepackage{pstricks}
\usepackage{pst-grad} % For gradients
\usepackage{pst-plot} % For axes

\begin{document}
\pagestyle{empty}

% Generated with LaTeXDraw 2.0.8
% Tue Dec 04 22:25:40 COT 2012
% \usepackage[usenames,dvipsnames]{pstricks}
% \usepackage{epsfig}
% \usepackage{pst-grad} % For gradients
% \usepackage{pst-plot} % For axes
\begin{pspicture}(0,-1.6491797)(3.4978907,1.6691797)
\psbezier[linewidth=0.04,arrowsize=0.05291667cm 2.0,arrowlength=1.4,arrowinset=0.4]{->}(0.037890624,0.43082032)(0.07789063,0.1908203)(2.8778906,0.6708203)(3.4778907,0.10297717)
\psbezier[linewidth=0.04,arrowsize=0.05291667cm 2.0,arrowlength=1.4,arrowinset=0.4]{->}(0.017890625,-0.08917969)(0.18966368,0.17082031)(2.7417204,0.030820312)(3.4778907,-0.17702283)
\psbezier[linewidth=0.04,arrowsize=0.05291667cm 2.0,arrowlength=1.4,arrowinset=0.4]{->}(0.057890624,-0.4291797)(0.09789062,-0.6091797)(2.6778905,-0.24917969)(3.4578905,-0.4291797)
\psdots[dotsize=0.12](1.6378906,-0.4291797)
\psline[linewidth=0.04cm](1.8778906,1.2708203)(1.4378906,-1.6291797)
\rput(1.9003711,1.4958203){$L$}
\rput(0.08578125,-0.6641797){\hspace{1cm}$x^0$}
\psdots[dotsize=0.12](0.23789063,-0.5091797)
\rput(1.9241211,-0.6041797){\hspace{1cm}$\phi(t,x^0)$}
\end{pspicture}

\end{document}